% Options for packages loaded elsewhere
\PassOptionsToPackage{unicode}{hyperref}
\PassOptionsToPackage{hyphens}{url}
%
\documentclass[
]{article}
\usepackage{amsmath,amssymb}
\usepackage{lmodern}
\usepackage{ifxetex,ifluatex}
\ifnum 0\ifxetex 1\fi\ifluatex 1\fi=0 % if pdftex
  \usepackage[T1]{fontenc}
  \usepackage[utf8]{inputenc}
  \usepackage{textcomp} % provide euro and other symbols
\else % if luatex or xetex
  \usepackage{unicode-math}
  \defaultfontfeatures{Scale=MatchLowercase}
  \defaultfontfeatures[\rmfamily]{Ligatures=TeX,Scale=1}
\fi
% Use upquote if available, for straight quotes in verbatim environments
\IfFileExists{upquote.sty}{\usepackage{upquote}}{}
\IfFileExists{microtype.sty}{% use microtype if available
  \usepackage[]{microtype}
  \UseMicrotypeSet[protrusion]{basicmath} % disable protrusion for tt fonts
}{}
\makeatletter
\@ifundefined{KOMAClassName}{% if non-KOMA class
  \IfFileExists{parskip.sty}{%
    \usepackage{parskip}
  }{% else
    \setlength{\parindent}{0pt}
    \setlength{\parskip}{6pt plus 2pt minus 1pt}}
}{% if KOMA class
  \KOMAoptions{parskip=half}}
\makeatother
\usepackage{xcolor}
\IfFileExists{xurl.sty}{\usepackage{xurl}}{} % add URL line breaks if available
\IfFileExists{bookmark.sty}{\usepackage{bookmark}}{\usepackage{hyperref}}
\hypersetup{
  pdftitle={Discussion\_10},
  pdfauthor={MGinorio},
  hidelinks,
  pdfcreator={LaTeX via pandoc}}
\urlstyle{same} % disable monospaced font for URLs
\usepackage[margin=1in]{geometry}
\usepackage{color}
\usepackage{fancyvrb}
\newcommand{\VerbBar}{|}
\newcommand{\VERB}{\Verb[commandchars=\\\{\}]}
\DefineVerbatimEnvironment{Highlighting}{Verbatim}{commandchars=\\\{\}}
% Add ',fontsize=\small' for more characters per line
\usepackage{framed}
\definecolor{shadecolor}{RGB}{248,248,248}
\newenvironment{Shaded}{\begin{snugshade}}{\end{snugshade}}
\newcommand{\AlertTok}[1]{\textcolor[rgb]{0.94,0.16,0.16}{#1}}
\newcommand{\AnnotationTok}[1]{\textcolor[rgb]{0.56,0.35,0.01}{\textbf{\textit{#1}}}}
\newcommand{\AttributeTok}[1]{\textcolor[rgb]{0.77,0.63,0.00}{#1}}
\newcommand{\BaseNTok}[1]{\textcolor[rgb]{0.00,0.00,0.81}{#1}}
\newcommand{\BuiltInTok}[1]{#1}
\newcommand{\CharTok}[1]{\textcolor[rgb]{0.31,0.60,0.02}{#1}}
\newcommand{\CommentTok}[1]{\textcolor[rgb]{0.56,0.35,0.01}{\textit{#1}}}
\newcommand{\CommentVarTok}[1]{\textcolor[rgb]{0.56,0.35,0.01}{\textbf{\textit{#1}}}}
\newcommand{\ConstantTok}[1]{\textcolor[rgb]{0.00,0.00,0.00}{#1}}
\newcommand{\ControlFlowTok}[1]{\textcolor[rgb]{0.13,0.29,0.53}{\textbf{#1}}}
\newcommand{\DataTypeTok}[1]{\textcolor[rgb]{0.13,0.29,0.53}{#1}}
\newcommand{\DecValTok}[1]{\textcolor[rgb]{0.00,0.00,0.81}{#1}}
\newcommand{\DocumentationTok}[1]{\textcolor[rgb]{0.56,0.35,0.01}{\textbf{\textit{#1}}}}
\newcommand{\ErrorTok}[1]{\textcolor[rgb]{0.64,0.00,0.00}{\textbf{#1}}}
\newcommand{\ExtensionTok}[1]{#1}
\newcommand{\FloatTok}[1]{\textcolor[rgb]{0.00,0.00,0.81}{#1}}
\newcommand{\FunctionTok}[1]{\textcolor[rgb]{0.00,0.00,0.00}{#1}}
\newcommand{\ImportTok}[1]{#1}
\newcommand{\InformationTok}[1]{\textcolor[rgb]{0.56,0.35,0.01}{\textbf{\textit{#1}}}}
\newcommand{\KeywordTok}[1]{\textcolor[rgb]{0.13,0.29,0.53}{\textbf{#1}}}
\newcommand{\NormalTok}[1]{#1}
\newcommand{\OperatorTok}[1]{\textcolor[rgb]{0.81,0.36,0.00}{\textbf{#1}}}
\newcommand{\OtherTok}[1]{\textcolor[rgb]{0.56,0.35,0.01}{#1}}
\newcommand{\PreprocessorTok}[1]{\textcolor[rgb]{0.56,0.35,0.01}{\textit{#1}}}
\newcommand{\RegionMarkerTok}[1]{#1}
\newcommand{\SpecialCharTok}[1]{\textcolor[rgb]{0.00,0.00,0.00}{#1}}
\newcommand{\SpecialStringTok}[1]{\textcolor[rgb]{0.31,0.60,0.02}{#1}}
\newcommand{\StringTok}[1]{\textcolor[rgb]{0.31,0.60,0.02}{#1}}
\newcommand{\VariableTok}[1]{\textcolor[rgb]{0.00,0.00,0.00}{#1}}
\newcommand{\VerbatimStringTok}[1]{\textcolor[rgb]{0.31,0.60,0.02}{#1}}
\newcommand{\WarningTok}[1]{\textcolor[rgb]{0.56,0.35,0.01}{\textbf{\textit{#1}}}}
\usepackage{graphicx}
\makeatletter
\def\maxwidth{\ifdim\Gin@nat@width>\linewidth\linewidth\else\Gin@nat@width\fi}
\def\maxheight{\ifdim\Gin@nat@height>\textheight\textheight\else\Gin@nat@height\fi}
\makeatother
% Scale images if necessary, so that they will not overflow the page
% margins by default, and it is still possible to overwrite the defaults
% using explicit options in \includegraphics[width, height, ...]{}
\setkeys{Gin}{width=\maxwidth,height=\maxheight,keepaspectratio}
% Set default figure placement to htbp
\makeatletter
\def\fps@figure{htbp}
\makeatother
\setlength{\emergencystretch}{3em} % prevent overfull lines
\providecommand{\tightlist}{%
  \setlength{\itemsep}{0pt}\setlength{\parskip}{0pt}}
\setcounter{secnumdepth}{-\maxdimen} % remove section numbering
\ifluatex
  \usepackage{selnolig}  % disable illegal ligatures
\fi

\title{Discussion\_10}
\author{MGinorio}
\date{10/30/2021}

\begin{document}
\maketitle

\hypertarget{example-11.6-pg.-410}{%
\subsection{Example 11.6 pg. 410}\label{example-11.6-pg.-410}}

In the Dark Ages, Harvard, Dartmouth, and Yale admitted only male
students. Assume that, at that time, 80 percent of the sons of Harvard
men went to Harvard and the rest went to Yale, 40 percent of the sons of
Yale men went to Yale, and the rest split evenly between Harvard and
Dartmouth; and of the sons of Dartmouth men, 70 percent went to
Dartmouth, 20 percent to Harvard, and 10 percent to Yale. We form a
Markov chain with transition matrix.

\$\$ P =

\begin{bmatrix}
    0.8 & 0 & 0.2 \\
    0.2 & 0.70 & 0.10 \\
    0.3 & 0 & 0.40 \\
    
\end{bmatrix}

\$\$ \#\# Step 1 Create vector College\_Admis containing the percentages
of college admission for each graduate.

\begin{Shaded}
\begin{Highlighting}[]
\NormalTok{College\_Zone }\OtherTok{\textless{}{-}} \FunctionTok{c}\NormalTok{(}\StringTok{"Harv"}\NormalTok{, }\StringTok{"Dartm"}\NormalTok{, }\StringTok{"Yale"}\NormalTok{)}
\NormalTok{College\_Zone}
\end{Highlighting}
\end{Shaded}

\begin{verbatim}
## [1] "Harv"  "Dartm" "Yale"
\end{verbatim}

\hypertarget{step-2}{%
\subsection{Step 2}\label{step-2}}

Create Matrix with values from probability. convert vector into matrix
and add to column and row name

\begin{Shaded}
\begin{Highlighting}[]
\NormalTok{Zone\_Transition }\OtherTok{=} \FunctionTok{matrix}\NormalTok{(}\FunctionTok{c}\NormalTok{(}\FloatTok{0.8}\NormalTok{,}\DecValTok{0}\NormalTok{,}\FloatTok{0.2}\NormalTok{,}\FloatTok{0.20}\NormalTok{,}\FloatTok{0.70}\NormalTok{,}\FloatTok{0.10}\NormalTok{,}\FloatTok{0.30}\NormalTok{,}\FloatTok{0.30}\NormalTok{,}\FloatTok{0.40}\NormalTok{), }
                         \AttributeTok{nrow =} \DecValTok{3}\NormalTok{,}
                         \AttributeTok{byrow =} \ConstantTok{TRUE}\NormalTok{,}
                         \AttributeTok{dimnames =} \FunctionTok{list}\NormalTok{(College\_Zone, College\_Zone))}
\NormalTok{Zone\_Transition}
\end{Highlighting}
\end{Shaded}

\begin{verbatim}
##       Harv Dartm Yale
## Harv   0.8   0.0  0.2
## Dartm  0.2   0.7  0.1
## Yale   0.3   0.3  0.4
\end{verbatim}

\hypertarget{step-3}{%
\subsection{Step 3}\label{step-3}}

Install markovchain package

\begin{Shaded}
\begin{Highlighting}[]
\FunctionTok{library}\NormalTok{(markovchain)}
\end{Highlighting}
\end{Shaded}

\begin{verbatim}
## Package:  markovchain
## Version:  0.8.6
## Date:     2021-05-17
## BugReport: https://github.com/spedygiorgio/markovchain/issues
\end{verbatim}

\hypertarget{step-4}{%
\subsection{Step 4}\label{step-4}}

Create a markov Chain object state space = to vector in Step 1 and
Transition matrix from step 2

\begin{Shaded}
\begin{Highlighting}[]
\NormalTok{mcZone }\OtherTok{\textless{}{-}} \FunctionTok{new}\NormalTok{(}\StringTok{"markovchain"}\NormalTok{, }\AttributeTok{states =}\NormalTok{ College\_Zone,}
               \AttributeTok{byrow =} \ConstantTok{TRUE}\NormalTok{,}
               \AttributeTok{transitionMatrix =}\NormalTok{ Zone\_Transition,}
               \AttributeTok{name =} \StringTok{"College\_Movement"}\NormalTok{)}
\NormalTok{mcZone}
\end{Highlighting}
\end{Shaded}

\begin{verbatim}
## College_Movement 
##  A  3 - dimensional discrete Markov Chain defined by the following states: 
##  Harv, Dartm, Yale 
##  The transition matrix  (by rows)  is defined as follows: 
##       Harv Dartm Yale
## Harv   0.8   0.0  0.2
## Dartm  0.2   0.7  0.1
## Yale   0.3   0.3  0.4
\end{verbatim}

\begin{Shaded}
\begin{Highlighting}[]
\FunctionTok{class}\NormalTok{(mcZone)}
\end{Highlighting}
\end{Shaded}

\begin{verbatim}
## [1] "markovchain"
## attr(,"package")
## [1] "markovchain"
\end{verbatim}

\hypertarget{step-5}{%
\subsection{Step 5}\label{step-5}}

For Example 11.6, find the probability that the grandson of a man from
Harvard went to Harvard

\begin{Shaded}
\begin{Highlighting}[]
\NormalTok{mcZone}\SpecialCharTok{\^{}}\DecValTok{2}
\end{Highlighting}
\end{Shaded}

\begin{verbatim}
## College_Movement^2 
##  A  3 - dimensional discrete Markov Chain defined by the following states: 
##  Harv, Dartm, Yale 
##  The transition matrix  (by rows)  is defined as follows: 
##       Harv Dartm Yale
## Harv  0.70  0.06 0.24
## Dartm 0.33  0.52 0.15
## Yale  0.42  0.33 0.25
\end{verbatim}

\hypertarget{answer}

\hypertarget{step-7}{%
\subsection{Step 7}\label{step-7}}

Determine the stationary state of

\begin{Shaded}
\begin{Highlighting}[]
\FunctionTok{steadyStates}\NormalTok{(mcZone)}
\end{Highlighting}
\end{Shaded}

\begin{verbatim}
##           Harv     Dartm      Yale
## [1,] 0.5555556 0.2222222 0.2222222
\end{verbatim}

\hypertarget{step-8}{%
\subsection{Step 8}\label{step-8}}

Display the Markov chain and the transition probabilities.

\begin{Shaded}
\begin{Highlighting}[]
\NormalTok{layout }\OtherTok{\textless{}{-}}\NormalTok{ Zone\_Transition}
\FunctionTok{plot}\NormalTok{(mcZone, }\AttributeTok{node.size =} \DecValTok{10}\NormalTok{, }\AttributeTok{layout =}\NormalTok{ layout)}
\end{Highlighting}
\end{Shaded}

\includegraphics{Discussion_10_files/figure-latex/unnamed-chunk-7-1.pdf}

Since we DO NOT have an absorbing Markov chain, we DO NOT calculate the
expected time until absorption.

\end{document}
