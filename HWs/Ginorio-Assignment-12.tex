% Options for packages loaded elsewhere
\PassOptionsToPackage{unicode}{hyperref}
\PassOptionsToPackage{hyphens}{url}
%
\documentclass[
]{article}
\usepackage{amsmath,amssymb}
\usepackage{lmodern}
\usepackage{ifxetex,ifluatex}
\ifnum 0\ifxetex 1\fi\ifluatex 1\fi=0 % if pdftex
  \usepackage[T1]{fontenc}
  \usepackage[utf8]{inputenc}
  \usepackage{textcomp} % provide euro and other symbols
\else % if luatex or xetex
  \usepackage{unicode-math}
  \defaultfontfeatures{Scale=MatchLowercase}
  \defaultfontfeatures[\rmfamily]{Ligatures=TeX,Scale=1}
\fi
% Use upquote if available, for straight quotes in verbatim environments
\IfFileExists{upquote.sty}{\usepackage{upquote}}{}
\IfFileExists{microtype.sty}{% use microtype if available
  \usepackage[]{microtype}
  \UseMicrotypeSet[protrusion]{basicmath} % disable protrusion for tt fonts
}{}
\makeatletter
\@ifundefined{KOMAClassName}{% if non-KOMA class
  \IfFileExists{parskip.sty}{%
    \usepackage{parskip}
  }{% else
    \setlength{\parindent}{0pt}
    \setlength{\parskip}{6pt plus 2pt minus 1pt}}
}{% if KOMA class
  \KOMAoptions{parskip=half}}
\makeatother
\usepackage{xcolor}
\IfFileExists{xurl.sty}{\usepackage{xurl}}{} % add URL line breaks if available
\IfFileExists{bookmark.sty}{\usepackage{bookmark}}{\usepackage{hyperref}}
\hypersetup{
  pdftitle={Linear Regression-1},
  pdfauthor={MGinorio},
  hidelinks,
  pdfcreator={LaTeX via pandoc}}
\urlstyle{same} % disable monospaced font for URLs
\usepackage[margin=1in]{geometry}
\usepackage{color}
\usepackage{fancyvrb}
\newcommand{\VerbBar}{|}
\newcommand{\VERB}{\Verb[commandchars=\\\{\}]}
\DefineVerbatimEnvironment{Highlighting}{Verbatim}{commandchars=\\\{\}}
% Add ',fontsize=\small' for more characters per line
\usepackage{framed}
\definecolor{shadecolor}{RGB}{248,248,248}
\newenvironment{Shaded}{\begin{snugshade}}{\end{snugshade}}
\newcommand{\AlertTok}[1]{\textcolor[rgb]{0.94,0.16,0.16}{#1}}
\newcommand{\AnnotationTok}[1]{\textcolor[rgb]{0.56,0.35,0.01}{\textbf{\textit{#1}}}}
\newcommand{\AttributeTok}[1]{\textcolor[rgb]{0.77,0.63,0.00}{#1}}
\newcommand{\BaseNTok}[1]{\textcolor[rgb]{0.00,0.00,0.81}{#1}}
\newcommand{\BuiltInTok}[1]{#1}
\newcommand{\CharTok}[1]{\textcolor[rgb]{0.31,0.60,0.02}{#1}}
\newcommand{\CommentTok}[1]{\textcolor[rgb]{0.56,0.35,0.01}{\textit{#1}}}
\newcommand{\CommentVarTok}[1]{\textcolor[rgb]{0.56,0.35,0.01}{\textbf{\textit{#1}}}}
\newcommand{\ConstantTok}[1]{\textcolor[rgb]{0.00,0.00,0.00}{#1}}
\newcommand{\ControlFlowTok}[1]{\textcolor[rgb]{0.13,0.29,0.53}{\textbf{#1}}}
\newcommand{\DataTypeTok}[1]{\textcolor[rgb]{0.13,0.29,0.53}{#1}}
\newcommand{\DecValTok}[1]{\textcolor[rgb]{0.00,0.00,0.81}{#1}}
\newcommand{\DocumentationTok}[1]{\textcolor[rgb]{0.56,0.35,0.01}{\textbf{\textit{#1}}}}
\newcommand{\ErrorTok}[1]{\textcolor[rgb]{0.64,0.00,0.00}{\textbf{#1}}}
\newcommand{\ExtensionTok}[1]{#1}
\newcommand{\FloatTok}[1]{\textcolor[rgb]{0.00,0.00,0.81}{#1}}
\newcommand{\FunctionTok}[1]{\textcolor[rgb]{0.00,0.00,0.00}{#1}}
\newcommand{\ImportTok}[1]{#1}
\newcommand{\InformationTok}[1]{\textcolor[rgb]{0.56,0.35,0.01}{\textbf{\textit{#1}}}}
\newcommand{\KeywordTok}[1]{\textcolor[rgb]{0.13,0.29,0.53}{\textbf{#1}}}
\newcommand{\NormalTok}[1]{#1}
\newcommand{\OperatorTok}[1]{\textcolor[rgb]{0.81,0.36,0.00}{\textbf{#1}}}
\newcommand{\OtherTok}[1]{\textcolor[rgb]{0.56,0.35,0.01}{#1}}
\newcommand{\PreprocessorTok}[1]{\textcolor[rgb]{0.56,0.35,0.01}{\textit{#1}}}
\newcommand{\RegionMarkerTok}[1]{#1}
\newcommand{\SpecialCharTok}[1]{\textcolor[rgb]{0.00,0.00,0.00}{#1}}
\newcommand{\SpecialStringTok}[1]{\textcolor[rgb]{0.31,0.60,0.02}{#1}}
\newcommand{\StringTok}[1]{\textcolor[rgb]{0.31,0.60,0.02}{#1}}
\newcommand{\VariableTok}[1]{\textcolor[rgb]{0.00,0.00,0.00}{#1}}
\newcommand{\VerbatimStringTok}[1]{\textcolor[rgb]{0.31,0.60,0.02}{#1}}
\newcommand{\WarningTok}[1]{\textcolor[rgb]{0.56,0.35,0.01}{\textbf{\textit{#1}}}}
\usepackage{longtable,booktabs,array}
\usepackage{calc} % for calculating minipage widths
% Correct order of tables after \paragraph or \subparagraph
\usepackage{etoolbox}
\makeatletter
\patchcmd\longtable{\par}{\if@noskipsec\mbox{}\fi\par}{}{}
\makeatother
% Allow footnotes in longtable head/foot
\IfFileExists{footnotehyper.sty}{\usepackage{footnotehyper}}{\usepackage{footnote}}
\makesavenoteenv{longtable}
\usepackage{graphicx}
\makeatletter
\def\maxwidth{\ifdim\Gin@nat@width>\linewidth\linewidth\else\Gin@nat@width\fi}
\def\maxheight{\ifdim\Gin@nat@height>\textheight\textheight\else\Gin@nat@height\fi}
\makeatother
% Scale images if necessary, so that they will not overflow the page
% margins by default, and it is still possible to overwrite the defaults
% using explicit options in \includegraphics[width, height, ...]{}
\setkeys{Gin}{width=\maxwidth,height=\maxheight,keepaspectratio}
% Set default figure placement to htbp
\makeatletter
\def\fps@figure{htbp}
\makeatother
\setlength{\emergencystretch}{3em} % prevent overfull lines
\providecommand{\tightlist}{%
  \setlength{\itemsep}{0pt}\setlength{\parskip}{0pt}}
\setcounter{secnumdepth}{-\maxdimen} % remove section numbering
\ifluatex
  \usepackage{selnolig}  % disable illegal ligatures
\fi

\title{Linear Regression-1}
\author{MGinorio}
\date{11/3/2021}

\begin{document}
\maketitle

\begin{Shaded}
\begin{Highlighting}[]
\FunctionTok{library}\NormalTok{(tidyverse)}
\FunctionTok{library}\NormalTok{(moderndive)}
\FunctionTok{library}\NormalTok{(skimr)}
\FunctionTok{library}\NormalTok{(ISLR)}
\FunctionTok{library}\NormalTok{(tinytex)}
\end{Highlighting}
\end{Shaded}

\hypertarget{dataset}{%
\subsection{Dataset}\label{dataset}}

\textbf{evals dataset}

Researchers at the University of Texas in Austin, Texas (UT Austin)
tried to answer the following research question: what factors explain
differences in instructor teaching evaluation scores?

To this end, they collected instructor and course information on 463
courses. A full description of the study can be found at openintro.org.

\begin{Shaded}
\begin{Highlighting}[]
\FunctionTok{glimpse}\NormalTok{(evals)}
\end{Highlighting}
\end{Shaded}

\begin{verbatim}
## Rows: 463
## Columns: 14
## $ ID           <int> 1, 2, 3, 4, 5, 6, 7, 8, 9, 10, 11, 12, 13, 14, 15, 16, 17~
## $ prof_ID      <int> 1, 1, 1, 1, 2, 2, 2, 3, 3, 4, 4, 4, 4, 4, 4, 4, 4, 5, 5, ~
## $ score        <dbl> 4.7, 4.1, 3.9, 4.8, 4.6, 4.3, 2.8, 4.1, 3.4, 4.5, 3.8, 4.~
## $ age          <int> 36, 36, 36, 36, 59, 59, 59, 51, 51, 40, 40, 40, 40, 40, 4~
## $ bty_avg      <dbl> 5.000, 5.000, 5.000, 5.000, 3.000, 3.000, 3.000, 3.333, 3~
## $ gender       <fct> female, female, female, female, male, male, male, male, m~
## $ ethnicity    <fct> minority, minority, minority, minority, not minority, not~
## $ language     <fct> english, english, english, english, english, english, eng~
## $ rank         <fct> tenure track, tenure track, tenure track, tenure track, t~
## $ pic_outfit   <fct> not formal, not formal, not formal, not formal, not forma~
## $ pic_color    <fct> color, color, color, color, color, color, color, color, c~
## $ cls_did_eval <int> 24, 86, 76, 77, 17, 35, 39, 55, 111, 40, 24, 24, 17, 14, ~
## $ cls_students <int> 43, 125, 125, 123, 20, 40, 44, 55, 195, 46, 27, 25, 20, 2~
## $ cls_level    <fct> upper, upper, upper, upper, upper, upper, upper, upper, u~
\end{verbatim}

\hypertarget{question}{%
\subsubsection{Question}\label{question}}

Can we explain differences in teaching evaluation score based on various
teacher attributes?

\hypertarget{variables}{%
\subsubsection{Variables}\label{variables}}

\(y:\) Average teaching \(score\) based on students evaluations

\(\vec{x}\) Attributes like gender, ethnicity, bty\_avg

Investigate correlation between this variables

\begin{itemize}
\tightlist
\item
  ID
\item
  Score
\item
  Age
\item
  Gender
\item
  Rank
\item
  Cls\_students
\end{itemize}

\begin{Shaded}
\begin{Highlighting}[]
\NormalTok{evals\_disc12 }\OtherTok{\textless{}{-}}\NormalTok{ evals }\SpecialCharTok{\%\textgreater{}\%} 
  \FunctionTok{select}\NormalTok{(ID, score, gender, age, rank, cls\_students, ethnicity)}

\NormalTok{evals\_disc12}
\end{Highlighting}
\end{Shaded}

\begin{verbatim}
## # A tibble: 463 x 7
##       ID score gender   age rank         cls_students ethnicity   
##    <int> <dbl> <fct>  <int> <fct>               <int> <fct>       
##  1     1   4.7 female    36 tenure track           43 minority    
##  2     2   4.1 female    36 tenure track          125 minority    
##  3     3   3.9 female    36 tenure track          125 minority    
##  4     4   4.8 female    36 tenure track          123 minority    
##  5     5   4.6 male      59 tenured                20 not minority
##  6     6   4.3 male      59 tenured                40 not minority
##  7     7   2.8 male      59 tenured                44 not minority
##  8     8   4.1 male      51 tenured                55 not minority
##  9     9   3.4 male      51 tenured               195 not minority
## 10    10   4.5 female    40 tenured                46 not minority
## # ... with 453 more rows
\end{verbatim}

\hypertarget{eda}{%
\subsubsection{EDA}\label{eda}}

Understanding Each Variable

We will perform an exploratory analysis of the selected variables before
any formal modeling

\includegraphics{Ginorio-Assignment-12_files/figure-latex/eda vars-1.pdf}

\hypertarget{statistics}{%
\subsubsection{Statistics}\label{statistics}}

\begin{Shaded}
\begin{Highlighting}[]
\NormalTok{evals\_disc12 }\SpecialCharTok{\%\textgreater{}\%} 
  \FunctionTok{select}\NormalTok{(score, gender, age, rank, cls\_students, ethnicity) }\SpecialCharTok{\%\textgreater{}\%} \FunctionTok{skim}\NormalTok{()}
\end{Highlighting}
\end{Shaded}

\begin{longtable}[]{@{}ll@{}}
\caption{Data summary}\tabularnewline
\toprule
& \\
\midrule
\endfirsthead
\toprule
& \\
\midrule
\endhead
Name & Piped data \\
Number of rows & 463 \\
Number of columns & 6 \\
\_\_\_\_\_\_\_\_\_\_\_\_\_\_\_\_\_\_\_\_\_\_\_ & \\
Column type frequency: & \\
factor & 3 \\
numeric & 3 \\
\_\_\_\_\_\_\_\_\_\_\_\_\_\_\_\_\_\_\_\_\_\_\_\_ & \\
Group variables & None \\
\bottomrule
\end{longtable}

\textbf{Variable type: factor}

\begin{longtable}[]{@{}lrrlrl@{}}
\toprule
skim\_variable & n\_missing & complete\_rate & ordered & n\_unique &
top\_counts \\
\midrule
\endhead
gender & 0 & 1 & FALSE & 2 & mal: 268, fem: 195 \\
rank & 0 & 1 & FALSE & 3 & ten: 253, ten: 108, tea: 102 \\
ethnicity & 0 & 1 & FALSE & 2 & not: 399, min: 64 \\
\bottomrule
\end{longtable}

\textbf{Variable type: numeric}

\begin{longtable}[]{@{}lrrrrrrrrrl@{}}
\toprule
skim\_variable & n\_missing & complete\_rate & mean & sd & p0 & p25 &
p50 & p75 & p100 & hist \\
\midrule
\endhead
score & 0 & 1 & 4.17 & 0.54 & 2.3 & 3.8 & 4.3 & 4.6 & 5 & ▁▁▅▇▇ \\
age & 0 & 1 & 48.37 & 9.80 & 29.0 & 42.0 & 48.0 & 57.0 & 73 & ▅▆▇▆▁ \\
cls\_students & 0 & 1 & 55.18 & 75.07 & 8.0 & 19.0 & 29.0 & 60.0 & 581 &
▇▁▁▁▁ \\
\bottomrule
\end{longtable}

\textbf{Visual Relationship}

Following the scatterplots indicate the relationship of the independent
variables witht the dependent variable (score)

\includegraphics{Ginorio-Assignment-12_files/figure-latex/unnamed-chunk-4-1.pdf}

\begin{itemize}
\tightlist
\item
  Score =
\item
  Age =
\item
  Cls\_students =
\end{itemize}

CATEGORICAL VARS -

When using a categorical predictor variable, the intercept corresponds
to the mean for the baseline group, while coefficients for the
non-baseline groups are offsets from this baseline. Thus in the
visualization the baseline for comparison group's median is marked with
a solid line, whereas all offset groups' medians are marked with dashed
lines

\begin{itemize}
\tightlist
\item
  Gender =
\item
  Rank =
\item
  Ethnicity =
\end{itemize}

\hypertarget{dichotomous-var}{%
\subsubsection{Dichotomous Var}\label{dichotomous-var}}

We can see in the scatter plot above that the amount of students in the
classrooms goes from 0-600. In which the vast majority is distributed
between 0-200. Having that in mind I will create a Dichotomous variable
for num\_students in a classroom less than 100 for ``Y'' and more than
100 to ``N''

\begin{Shaded}
\begin{Highlighting}[]
\NormalTok{evals\_disc12}\SpecialCharTok{$}\NormalTok{num\_students }\OtherTok{\textless{}{-}} \FunctionTok{ifelse}\NormalTok{(evals\_disc12}\SpecialCharTok{$}\NormalTok{cls\_students }\SpecialCharTok{\textless{}=} \DecValTok{100}\NormalTok{, }\StringTok{"Y"}\NormalTok{, }\StringTok{"N"}\NormalTok{)}
\NormalTok{evals\_disc12}\SpecialCharTok{$}\NormalTok{num\_students\_n }\OtherTok{\textless{}{-}} \FunctionTok{ifelse}\NormalTok{(evals\_disc12}\SpecialCharTok{$}\NormalTok{num\_students }\SpecialCharTok{==} \StringTok{"Y"}\NormalTok{, }\DecValTok{1}\NormalTok{,}\DecValTok{0}\NormalTok{)}
\end{Highlighting}
\end{Shaded}

\begin{Shaded}
\begin{Highlighting}[]
\NormalTok{evals\_disc12 }\SpecialCharTok{\%\textgreater{}\%} 
  \FunctionTok{ggplot}\NormalTok{(}\FunctionTok{aes}\NormalTok{(}\AttributeTok{x =}\NormalTok{ num\_students, }\AttributeTok{y =}\NormalTok{ score)) }\SpecialCharTok{+}
  \FunctionTok{geom\_boxplot}\NormalTok{() }\SpecialCharTok{+} 
  \FunctionTok{labs}\NormalTok{(}\AttributeTok{title =} \StringTok{"num\_students vs. score"}\NormalTok{)}
\end{Highlighting}
\end{Shaded}

\includegraphics{Ginorio-Assignment-12_files/figure-latex/unnamed-chunk-6-1.pdf}
num\_students boxplot indicates a slight difference in score when
teacher has more than 100 students

\hypertarget{create-model}{%
\subsubsection{Create Model}\label{create-model}}

\textbf{Score Model}

\begin{verbatim}
## 
## Call:
## lm(formula = score ~ gender + age + rank + cls_students + ethnicity, 
##     data = evals_disc12)
## 
## Residuals:
##     Min      1Q  Median      3Q     Max 
## -1.7769 -0.3589  0.0773  0.4233  1.0083 
## 
## Coefficients:
##                         Estimate Std. Error t value Pr(>|t|)    
## (Intercept)            4.649e+00  1.759e-01  26.429  < 2e-16 ***
## gendermale             1.942e-01  5.328e-02   3.645 0.000298 ***
## age                   -1.098e-02  3.111e-03  -3.531 0.000457 ***
## ranktenure track      -2.165e-01  8.246e-02  -2.626 0.008941 ** 
## ranktenured           -1.659e-01  6.395e-02  -2.594 0.009794 ** 
## cls_students           8.954e-05  3.380e-04   0.265 0.791216    
## ethnicitynot minority  9.342e-02  7.318e-02   1.277 0.202406    
## ---
## Signif. codes:  0 '***' 0.001 '**' 0.01 '*' 0.05 '.' 0.1 ' ' 1
## 
## Residual standard error: 0.5299 on 456 degrees of freedom
## Multiple R-squared:  0.06308,    Adjusted R-squared:  0.05076 
## F-statistic: 5.117 on 6 and 456 DF,  p-value: 4.21e-05
\end{verbatim}

\begin{Shaded}
\begin{Highlighting}[]
\FunctionTok{plot}\NormalTok{(score\_model)}
\end{Highlighting}
\end{Shaded}

\includegraphics{Ginorio-Assignment-12_files/figure-latex/unnamed-chunk-7-1.pdf}
\includegraphics{Ginorio-Assignment-12_files/figure-latex/unnamed-chunk-7-2.pdf}
\includegraphics{Ginorio-Assignment-12_files/figure-latex/unnamed-chunk-7-3.pdf}
\includegraphics{Ginorio-Assignment-12_files/figure-latex/unnamed-chunk-7-4.pdf}

\hypertarget{transform-variables}{%
\subsubsection{Transform Variables}\label{transform-variables}}

\textbf{Quadratic Term}

I decided to create quadratic term for class students

\begin{Shaded}
\begin{Highlighting}[]
\NormalTok{evals\_disc12}\SpecialCharTok{$}\NormalTok{cls\_students\_sq2 }\OtherTok{\textless{}{-}}\NormalTok{ evals\_disc12}\SpecialCharTok{$}\NormalTok{cls\_students}\SpecialCharTok{\^{}}\DecValTok{2}
\end{Highlighting}
\end{Shaded}

\textbf{Dichonomous by quatitative}

to create a dichotomous by quatitative variable. I will multiply age by
a Dichotomous variable `num\_students'

num\_students was created earlier based on the scatterplot

\begin{Shaded}
\begin{Highlighting}[]
\NormalTok{evals\_disc12}\SpecialCharTok{$}\NormalTok{cls\_by\_age }\OtherTok{\textless{}{-}}\NormalTok{ evals\_disc12}\SpecialCharTok{$}\NormalTok{num\_students\_n }\SpecialCharTok{*}\NormalTok{ evals\_disc12}\SpecialCharTok{$}\NormalTok{age}
\end{Highlighting}
\end{Shaded}

\hypertarget{second-model}{%
\subsubsection{Second Model}\label{second-model}}

\textbf{New Variables}

\begin{itemize}
\item
  cls\_by\_age
\item
  num\_students\_n
\item
  num\_students
\end{itemize}

\begin{Shaded}
\begin{Highlighting}[]
\NormalTok{score\_model\_t }\OtherTok{\textless{}{-}} \FunctionTok{lm}\NormalTok{(score }\SpecialCharTok{\textasciitilde{}}\NormalTok{ cls\_by\_age }\SpecialCharTok{+}\NormalTok{ num\_students }\SpecialCharTok{+}\NormalTok{ cls\_students\_sq2, }\AttributeTok{data =}\NormalTok{ evals\_disc12)}

\FunctionTok{summary}\NormalTok{(score\_model\_t)}
\end{Highlighting}
\end{Shaded}

\begin{verbatim}
## 
## Call:
## lm(formula = score ~ cls_by_age + num_students + cls_students_sq2, 
##     data = evals_disc12)
## 
## Residuals:
##     Min      1Q  Median      3Q     Max 
## -1.9245 -0.3462  0.1093  0.4183  0.8889 
## 
## Coefficients:
##                    Estimate Std. Error t value Pr(>|t|)    
## (Intercept)       4.022e+00  8.643e-02  46.530   <2e-16 ***
## cls_by_age       -6.674e-03  2.777e-03  -2.403   0.0166 *  
## num_studentsY     4.761e-01  1.612e-01   2.954   0.0033 ** 
## cls_students_sq2  2.205e-06  8.547e-07   2.580   0.0102 *  
## ---
## Signif. codes:  0 '***' 0.001 '**' 0.01 '*' 0.05 '.' 0.1 ' ' 1
## 
## Residual standard error: 0.5382 on 459 degrees of freedom
## Multiple R-squared:  0.02692,    Adjusted R-squared:  0.02056 
## F-statistic: 4.232 on 3 and 459 DF,  p-value: 0.005752
\end{verbatim}

\begin{Shaded}
\begin{Highlighting}[]
\FunctionTok{plot}\NormalTok{(score\_model\_t)}
\end{Highlighting}
\end{Shaded}

\includegraphics{Ginorio-Assignment-12_files/figure-latex/unnamed-chunk-11-1.pdf}
\includegraphics{Ginorio-Assignment-12_files/figure-latex/unnamed-chunk-11-2.pdf}
\includegraphics{Ginorio-Assignment-12_files/figure-latex/unnamed-chunk-11-3.pdf}
\includegraphics{Ginorio-Assignment-12_files/figure-latex/unnamed-chunk-11-4.pdf}

\hypertarget{conclusion}{%
\subsection{Conclusion}\label{conclusion}}

Overall, Adjusted R-squared: 0.009235 wich means the model only accounts
for 0.9\% of the variability in the data.

the selection of class size (cls\_students) given the influence of the
outliers.

this model does not meet the level of the baseline model.

\end{document}
