% Options for packages loaded elsewhere
\PassOptionsToPackage{unicode}{hyperref}
\PassOptionsToPackage{hyphens}{url}
%
\documentclass[
]{article}
\usepackage{amsmath,amssymb}
\usepackage{lmodern}
\usepackage{ifxetex,ifluatex}
\ifnum 0\ifxetex 1\fi\ifluatex 1\fi=0 % if pdftex
  \usepackage[T1]{fontenc}
  \usepackage[utf8]{inputenc}
  \usepackage{textcomp} % provide euro and other symbols
\else % if luatex or xetex
  \usepackage{unicode-math}
  \defaultfontfeatures{Scale=MatchLowercase}
  \defaultfontfeatures[\rmfamily]{Ligatures=TeX,Scale=1}
\fi
% Use upquote if available, for straight quotes in verbatim environments
\IfFileExists{upquote.sty}{\usepackage{upquote}}{}
\IfFileExists{microtype.sty}{% use microtype if available
  \usepackage[]{microtype}
  \UseMicrotypeSet[protrusion]{basicmath} % disable protrusion for tt fonts
}{}
\makeatletter
\@ifundefined{KOMAClassName}{% if non-KOMA class
  \IfFileExists{parskip.sty}{%
    \usepackage{parskip}
  }{% else
    \setlength{\parindent}{0pt}
    \setlength{\parskip}{6pt plus 2pt minus 1pt}}
}{% if KOMA class
  \KOMAoptions{parskip=half}}
\makeatother
\usepackage{xcolor}
\IfFileExists{xurl.sty}{\usepackage{xurl}}{} % add URL line breaks if available
\IfFileExists{bookmark.sty}{\usepackage{bookmark}}{\usepackage{hyperref}}
\hypersetup{
  pdftitle={MAGinorio\_Assignment\_3},
  pdfauthor={MGinorio},
  hidelinks,
  pdfcreator={LaTeX via pandoc}}
\urlstyle{same} % disable monospaced font for URLs
\usepackage[margin=1in]{geometry}
\usepackage{graphicx}
\makeatletter
\def\maxwidth{\ifdim\Gin@nat@width>\linewidth\linewidth\else\Gin@nat@width\fi}
\def\maxheight{\ifdim\Gin@nat@height>\textheight\textheight\else\Gin@nat@height\fi}
\makeatother
% Scale images if necessary, so that they will not overflow the page
% margins by default, and it is still possible to overwrite the defaults
% using explicit options in \includegraphics[width, height, ...]{}
\setkeys{Gin}{width=\maxwidth,height=\maxheight,keepaspectratio}
% Set default figure placement to htbp
\makeatletter
\def\fps@figure{htbp}
\makeatother
\setlength{\emergencystretch}{3em} % prevent overfull lines
\providecommand{\tightlist}{%
  \setlength{\itemsep}{0pt}\setlength{\parskip}{0pt}}
\setcounter{secnumdepth}{-\maxdimen} % remove section numbering
\ifluatex
  \usepackage{selnolig}  % disable illegal ligatures
\fi

\title{MAGinorio\_Assignment\_3}
\author{MGinorio}
\date{9/9/2021}

\begin{document}
\maketitle

\hypertarget{problem-set-1}{%
\section{Problem Set 1}\label{problem-set-1}}

\hypertarget{what-is-the-rank-of-the-matrix-a}{%
\subsubsection{1. What is the rank of the Matrix
A?}\label{what-is-the-rank-of-the-matrix-a}}

\[
A = 
\begin{bmatrix}
    1 & 2 & 3 & 4 \\
    -1 & 0 & 1 & 3 \\
    0 & 1 & -2 & 1 \\
    5 & 4 & -2 & -3 \\
\end{bmatrix}
\] {[},1{]} {[},2{]} {[},3{]} {[},4{]} {[}1,{]} 1 2 3 4 {[}2,{]} -1 0 1
3 {[}3,{]} 0 1 -2 1 {[}4,{]} 5 4 -2 -3

\begin{verbatim}
## Warning: package 'pracma' was built under R version 4.0.5
\end{verbatim}

{[}1{]} 4

Matrix Rank is the \# of nonzero rows in the row echelon form of the
Matrix thus:

\begin{verbatim}
 [,1] [,2] [,3] [,4]
\end{verbatim}

{[}1,{]} 1 0 0 0 {[}2,{]} 0 1 0 0 {[}3,{]} 0 0 1 0 {[}4,{]} 0 0 0 1

\hypertarget{given-an-mxn-matrix-where-m-n-what-can-be-the-maximum-rank-the-minimum-rank-assuming-that-the-matrix-is-non-zero}{%
\subsubsection{2.- Given an mxn matrix where m \textgreater{} n, what
can be the maximum rank? The minimum rank, assuming that the matrix is
non-zero?}\label{given-an-mxn-matrix-where-m-n-what-can-be-the-maximum-rank-the-minimum-rank-assuming-that-the-matrix-is-non-zero}}

Maximum rank = n Minimum rank possible = 1 ( assuming non zero)

\hypertarget{what-is-the-rank-of-matrix-b}{%
\subsubsection{3 What is the rank of matrix
B?}\label{what-is-the-rank-of-matrix-b}}

\$\$ B =

\begin{bmatrix}
    1 & 2 & 1 \\
    3 & 6 & 3 \\
    2 & 4 & 2 \\
  
\end{bmatrix}

\$\$

\begin{verbatim}
 [,1] [,2] [,3]
\end{verbatim}

{[}1,{]} 1 2 1 {[}2,{]} 3 6 3 {[}3,{]} 2 4 2 {[}1{]} 1 {[},1{]} {[},2{]}
{[},3{]} {[}1,{]} 1 2 1 {[}2,{]} 0 0 0 {[}3,{]} 0 0 0

\hypertarget{problem-set-2}{%
\section{Problem Set 2}\label{problem-set-2}}

\hypertarget{compute-the-eigenvalues-and-eigenvectors-of-the-matrix-a.-youll-need-to-show-your-work.-youll-need-to-write-out-the-characteristic-polynomial-and-show-your-solution.}{%
\subsubsection{1.- Compute the eigenvalues and eigenvectors of the
matrix A. You'll need to show your work. You'll need to write out the
characteristic polynomial and show your
solution.}\label{compute-the-eigenvalues-and-eigenvectors-of-the-matrix-a.-youll-need-to-show-your-work.-youll-need-to-write-out-the-characteristic-polynomial-and-show-your-solution.}}

\$\$ C =

\begin{bmatrix}
    1 & 2 & 3 \\
    0 & 4 & 5 \\
    0 & 0 & 6 \\
  
\end{bmatrix}

\$\$ {[},1{]} {[},2{]} {[},3{]} {[}1,{]} 1 0 0 {[}2,{]} 2 4 0 {[}3,{]} 3
5 6

\begin{verbatim}
 [,1] [,2] [,3]
\end{verbatim}

{[}1,{]} 0 1 0 {[}2,{]} 0 0 1 {[}3,{]} 0 0 0 {[},1{]} {[},2{]} {[},3{]}
{[}1,{]} 1 0 -1.6 {[}2,{]} 0 1 -2.5 {[}3,{]} 0 0 0.0 {[},1{]} {[},2{]}
{[},3{]} {[}1,{]} 1 0 -1.6 {[}2,{]} 0 1 -2.5 {[}3,{]} 0 0 0.0 \$\$

Basis / Vector Space =

\begin{bmatrix}
    1  \\
    0  \\
    0 \\
  
\end{bmatrix}

\begin{bmatrix}
    -0.66  \\
    0  \\
    0 \\
  
\end{bmatrix}

\begin{bmatrix}
    0  \\
    -2.5\\
    1 \\
  
\end{bmatrix}

\$\$

\end{document}
